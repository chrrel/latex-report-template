\documentclass[12pt, a4paper, final]{report}

\newcommand{\isPrintVersion}{false}						% false -> show colored links, true -> all links in black

\usepackage[utf8]{inputenc}
%%%%%%%%%%%%%%%%%%%%%%%%%%%%%%%%%%%%%%%%%%%%%%%%%%%%%%%%%%%%%%%%%%%%%%%%%%%%%%%%
%%%%%%%%%%%%%%%%%%%%%%%%%%%%%%% Data  %%%%%%%%%%%%%%%%%%%%%%%%%%%%%%%%%%%%%%%%%%
\newcommand{\thesisTitle}{Analysis of Alien Life Forms and Their Relationship to Cats in Television Series}
\newcommand{\authorName}{Gordon Shumway}
\newcommand{\thesisDate}{$19^{th}$ Januar 2038}
\newcommand{\signatureDate}{19. Januar 2038}
\newcommand{\city}{Los Angeles}

%%%%% Title page
\newcommand{\reportType}{Project Report}
\newcommand{\courseOfStudies}{Applied Unicorn Science}
\newcommand{\university}{Andromeda University of Melmac}

\newcommand{\timeOfProject}{01/1970 -- 01/2038}
\newcommand{\studentId}{1234567}
\newcommand{\course}{XYZ1234}
\newcommand{\company}{MyCompany}
\newcommand{\supervisorName}{William Tanner}

\newcommand{\byTitle}{by}
\newcommand{\courseOfStudiesTitle}{Course of Studies}
\newcommand{\timeOfProjectTitle}{Time of Project}
\newcommand{\studentIdTitle}{Student Number}
\newcommand{\courseTitle}{Course}
\newcommand{\companyTitle}{Company}
\newcommand{\supervisorTitle}{Supervisor}

\newcommand{\universityLogo}{example-image-16x9.pdf}
\newcommand{\companyLogo}{example-image-16x9.pdf}

%%%%% Quote
\newcommand{\quoteText}{Designed to make a difference.}
\newcommand{\quoteAuthor}{The Minus Sign}
\newcommand{\quoteSource}{A Funny Algebra Book}

%%%%% Declaration
\newcommand{\declaration}{Ich versichere hiermit, dass ich meine Projektarbeit mit dem Thema \textit{\thesisTitle} \linebreak selbstständig verfasst und keine anderen als die angegebenen Quellen und \linebreak Hilfsmittel benutzt habe.}

\newcommand{\declarationTitle}{Ehrenwörtliche Erklärung}
\newcommand{\ListOfFiguresTitle}{List of Figures}
\newcommand{\ListOfAbbreviationsTitle}{List of Abbreviations}
\newcommand{\ListOfListingsTitle}{List of Listings}

%%%%% Document
\title{\thesisTitle}
\author{\authorName}

%%%%%%%%%%%%%%%%%%%%%%%%%%%%%%%%%%%%%%%%%%%%%%%%%%%%%%%%%%%%%%%%%%%%%%%%%%%%%%%%
%%%%%%%%%%%%%%%%%%%%%%%%%%%%%%%% Abstract  %%%%%%%%%%%%%%%%%%%%%%%%%%%%%%%%%%%%%
\usepackage{abstract}
\renewcommand{\abstractnamefont}{\large\bfseries}

\newcommand{\abstractText}{Lorem ipsum dolor sit amet, consectetuer adipiscing elit. Aenean commodo ligula eget dolor. Aenean massa. Cum sociis natoque penatibus et magnis dis parturient montes, nascetur ridiculus mus. Donec quam felis, ultricies nec, pellentesque eu, pretium quis, vulputate eget, arcu.

In enim justo, rhoncus ut, imperdiet a, venenatis vitae, justo. Nullam dictum felis eu pede mollis pretium. Integer tincidunt. Cras dapibus. Vivamus elementum semper nisi. Aenean vulputate eleifend tellus. Aenean leo ligula, porttitor eu, consequat vitae, feugiat a, tellus.

Phasellus viverra nulla ut metus varius laoreet. Quisque rutrum. Aenean imperdiet. Etiam ultricies nisi vel augue. Curabitur ullamcorper ultricies nisi. Nam eget dui. Etiam rhoncus. Maecenas tempus, tellus eget condimentum rhoncus, sem quam semper libero, sit amet adip leo. Sed fringilla mauris sit amet nibh. Donec sodales sagittis magna. Sed consequat, leo eget bibendum sodales.
}

\newcommand{\abstractGermanText}{Dies ist ein Typoblindtext. An ihm kann man sehen, ob alle Buchstaben da sind und wie sie aussehen. Manchmal benutzt man Worte wie Hamburgefonts, Rafgenduks oder Handgloves, um Schriften zu testen. Manchmal Sätze, die alle Buchstaben des Alphabets enthalten - man nennt diese Sätze Pangrams. Sehr bekannt ist dieser: The quick brown fox jumps over the lazy old dog.

 Oft werden in Typoblindtexte auch fremdsprachige Satzteile eingebaut (AVAIL® and Wefox™ are testing aussi la Kerning), um die Wirkung in anderen Sprachen zu testen. In Lateinisch sieht zum Beispiel fast jede Schrift gut aus. Quod erat demonstrandum. Seit 1975 fehlen in den meisten Testtexten die Zahlen, weswegen ab dem Jahr 2034 Zahlen in 86 der Texte zur Pflicht werden.

Nichteinhaltung wird mit bis zu 245 Euro oder 368 bestraft. Genauso wichtig in sind mittlerweile auch Âçcèñtë, die in neueren Schriften aber fast immer enthalten sind. Ein wichtiges aber schwierig zu integrierendes Feld sind OpenType-Funktionalitäten. Je nach Software und Voreinstellungen können eingebaute Kapitälchen, Kerning oder Ligaturen (sehr pfiffig) nicht richtig dargestellt werden.
}

%%%%%%%%%%%%%%%%%%%%%%%%%%%%%%%%%%%%%%%%%%%%%%%%%%%%%%%%%%%%%%%%%%%%%%%%%%%%%%%%
%%%%%%%%%%%%%%%%%%%%%%%%%%%%%%%% Packages  %%%%%%%%%%%%%%%%%%%%%%%%%%%%%%%%%%%%%
\usepackage[left=3.5cm,right=2.5cm,top=3cm,bottom=2.5cm]{geometry}
\usepackage[ngerman,english]{babel}
\usepackage{amsmath}
\usepackage{amsfonts}
\usepackage{amssymb}
\usepackage{enumitem}
\usepackage{float}
\usepackage{fancyhdr}
\usepackage[bottom]{footmisc} 							% stick footnotes to bottom of the page
\usepackage[printonlyused]{acronym}
\usepackage{appendix}
\usepackage{pdfpages}								% include PDF files
\usepackage{ifthen}
\usepackage{csquotes}

%%%%%%%%%%%%%%%%%%%%%%%%%%%%%%%%%%%%%%%%%%%%%%%%%%%%%%%%%%%%%%%%%%%%%%%%%%%%%%%%
%%%%%%%%%%%%%%%%%%%%%%%%%%%%%% Colors  %%%%%%%%%%%%%%%%%%%%%%%%%%%%%%%%%%%%%%%%%
\definecolor{darkBlue}{HTML}{043562}
\definecolor{lightBlue}{HTML}{6699cc}
\definecolor{darkRed}{HTML}{7F1913}
\definecolor{darkGrey}{HTML}{111111}
\definecolor{lightGrey}{HTML}{F5F7F9}
\definecolor{magenta}{HTML}{B10059}
\definecolor{gray75}{gray}{0.75}

%%%%%%%%%%%%%%%%%%%%%%%%%%%%%%%%%%%%%%%%%%%%%%%%%%%%%%%%%%%%%%%%%%%%%%%%%%%%%%%%
%%%%%%%%%%%%%%%%%%%%%%%%%%%%%% New commands  %%%%%%%%%%%%%%%%%%%%%%%%%%%%%%%%%%% 

%%%%% Caption with sources
\usepackage{caption}
\captionsetup{justification=centering}
\newcommand{\captionsource}[2][]{
	\ifthenelse{\equal{#1}{}}
		{\small\textit{Source: #2}} % if empty
		{\small\textit{Source: #2, \href{#1}{#1}} % else
	}
}

%%%%% Theorems / Definitions
\usepackage{amsthm} 					
\newtheorem{define}{Definition}

%%%%% Code in text
\newcommand{\code}[1]{{\ttfamily#1}}

%%%%%%%%%%%%%%%%%%%%%%%%%%%%%%%%%%%%%%%%%%%%%%%%%%%%%%%%%%%%%%%%%%%%%%%%%%%%%%%%
%%%%%%%%%%%%%%%%%%%%%%%%%%%%%% Bibliography  %%%%%%%%%%%%%%%%%%%%%%%%%%%%%%%%%%%
\usepackage[backend=bibtex,style=numeric,citestyle=numeric]{biblatex}
\bibliography{bibliography} 

% linebreaks in bibliography urls
\usepackage{url}
\usepackage{breakurl}
\def\UrlBreaks{\do\/\do-}							% add a breakpoint at - and /

\setcounter{biburlnumpenalty}{1000}  						% add a breakpoint after all numbers
\setcounter{biburlucpenalty}{1000}						% add a breakpoint after all uppercase letters
\setcounter{biburllcpenalty}{1000}   						% add a breakpoint after all lowercase letters

%%%%%%%%%%%%%%%%%%%%%%%%%%%%%%%%%%%%%%%%%%%%%%%%%%%%%%%%%%%%%%%%%%%%%%%%%%%%%%%%
%%%%%%%%%%%%%%%%%%%%%%%%%%%%%%% PDF settings  %%%%%%%%%%%%%%%%%%%%%%%%%%%%%%%%%%
\usepackage[pdftitle={\thesisTitle}, pdfauthor={\authorName},pdfsubject={\thesisTitle}, pdfcreator={pdflatex}, pdfpagemode=UseOutlines, pdfdisplaydoctitle=true, pdflang={en}, breaklinks]{hyperref}

% color settings for links in pdf
\ifthenelse{\equal{\isPrintVersion}{true}}
{ % print version
	\hypersetup{
		hidelinks=true,
		colorlinks=false, 
		linktocpage=true, 						% page numbers are clickable
		bookmarksnumbered=true 						% show heading numbering in pdf table of contents
	}
}{ % screen version
	\hypersetup{
		colorlinks=true, 
		linkcolor=darkBlue,
		citecolor=darkBlue,
		filecolor=darkBlue,
		menucolor=darkBlue,
		urlcolor=darkBlue,
		linktocpage=true, 						% page numbers are clickable
		bookmarksnumbered=true 						% show heading numbering in pdf table of contents
	}
}

%%%%%%%%%%%%%%%%%%%%%%%%%%%%%%%%%%%%%%%%%%%%%%%%%%%%%%%%%%%%%%%%%%%%%%%%%%%%%%%%
%%%%%%%%%%%%%%%%%%%%%%%%%%%%%%% Typography  %%%%%%%%%%%%%%%%%%%%%%%%%%%%%%%%%%%%
\usepackage{lmodern}
\usepackage{mathpazo}
\fontfamily{ppl}\selectfont

\setlength{\parskip}{8pt}							% after paragraph
\setlength{\parindent}{0pt}							% new paragraph indent
\renewcommand{\baselinestretch}{1.2}\normalsize					% line height (line height for content is defined in line 281)

% Prevent widows and orphans
\clubpenalty = 10000								% prevent orphans
\widowpenalty = 10000 								% prevent widows
\displaywidowpenalty=10000

% Headings
\usepackage{titlesec}
\newcommand{\hsp}{\hspace{20pt}}

\titleformat{\chapter}[block]{\Huge\bfseries}{\thechapter \hsp {|}\hsp}{0pt}{\Huge\bfseries}
\titleformat{\section}[block]{\Large\bfseries}{\thesection \ }{0pt}{\Large \bfseries}
\titleformat{\subsection}[block]{\large\bfseries}{\thesubsection \ }{0pt}{\large \bfseries}
%%%%%%%%%%%%%%%%%%%%%%%%%%%%%%%%%%%%%%%%%%%%%%%%%%%%%%%%%%%%%%%%%%%%%%%%%%%%%%%%
%%%%%%%%%%%%%%%%%%%%%%%%%%%%%%%%%%% Listings  %%%%%%%%%%%%%%%%%%%%%%%%%%%%%%%%%%
\usepackage{listings}

\lstset{
	language=Java,								% default language
	numbers=left,								% line number position, alternative: none
	stepnumber=1,								 
	numbersep=6pt,								% distance between line number and code
	numberstyle=\fontsize{9pt}{9pt}\ttfamily,				% line number style
	breaklines=true,							% break lines if necessary
	breakautoindent=true,							% indent after line break
	postbreak=\space,							% break at space
	tabsize=4,								% tab size
	showspaces=false,							% do not show spaces in code
	showstringspaces=false,							% do not show spaces in strings
	extendedchars=true,							% use Latin1
	captionpos=b,								% caption position
	backgroundcolor=\color{lightGrey}, 					% background color
	xleftmargin=0pt,								
	xrightmargin=0pt,							
	frame=leftline,								% draw frame ar left side only
	framerule=1.1pt,							% frame width
	frameround=ffff,							% round corners: f = not rounded | t = rounded
	rulecolor=\color{lightBlue},						% frame color
	framesep=1.5pt,								% distance between line number and border 
	fillcolor=\color{lightGrey},
	basicstyle=\ttfamily\footnotesize\color{darkGrey},	
	keywordstyle=\color{magenta}\bfseries,
	identifierstyle=,
	commentstyle=\color{lightBlue},
	stringstyle=\color{darkBlue}
}

\lstloadlanguages{bash, C, C++, HTML, Java, PHP, Python, SQL}

% change heading above listings to \ListOfListingsTitle
\renewcommand{\lstlistlistingname}{\ListOfListingsTitle}

% Group list of listings by chapter
\let\Chapter\chapter
\def\chapter{\addtocontents{lol}{\protect\addvspace{10pt}}\Chapter}

%%%%%%%%%%%%%%%%%%%%%%%%%%%%%%%%%%%%%%%%%%%%%%%%%%%%%%%%%%%%%%%%%%%%%%%%%%%%%%%%
%%%%%%%%%%%%%%%%%%%%%%%%%%%%%%%%%%% Diagrams  %%%%%%%%%%%%%%%%%%%%%%%%%%%%%%%%%%
\usepackage{tikz}
\usetikzlibrary{shapes,arrows,shadows}
% flow diagrams
\tikzstyle{startstop} = [rectangle, rounded corners, minimum width=3cm, minimum height=0.9cm,text centered, draw=lightBlue, fill=lightGrey]
\tikzstyle{io} = [trapezium, trapezium left angle=70, trapezium right angle=110, minimum width=3cm, minimum height=0.9cm, text centered, draw=lightBlue, fill=lightGrey]
\tikzstyle{process} = [rectangle, minimum width=3cm, minimum height=0.9cm, text centered, text width=3cm, draw=lightBlue, fill=lightGrey]
\tikzstyle{decision} = [diamond, minimum width=3cm, minimum height=0.9cm, text centered, draw=lightBlue, fill=lightGrey]
\tikzstyle{arrow} = [thick,->,>=stealth]
\tikzstyle{comment} = [rectangle, minimum width=1cm, minimum height=0.9cm, text centered, text width=3cm, draw=none, fill=none]
\tikzstyle{subprocess} = [rectangle split, rectangle split horizontal, rectangle split parts=3, minimum height=0.9cm, minimum width=3cm, draw=lightBlue, fill=lightGrey]

% sequence diagrams
\usepackage[underline=false]{pgf-umlsd}

%%%%%%%%%%%%%%%%%%%%%%%%%%%%%%%%%%%%%%%%%%%%%%%%%%%%%%%%%%%%%%%%%%%%%%%%%%%%%%%%
%%%%%%%%%%%%%%%%%%%%%%%%%%%%% Todo Comments  %%%%%%%%%%%%%%%%%%%%%%%%%%%%%%%%%%%
\usepackage[colorinlistoftodos]{todonotes}
\reversemarginpar
\setlength{\marginparwidth}{3cm}

\newcommand{\todoQuestion}[1]{\todo[color=green!40]{#1}}
\newcommand{\todoImportant}[1]{\todo[color=red!60]{#1}}

%%%%%%%%%%%%%%%%%%%%%%%%%%%%%%%%%%%%%%%%%%%%%%%%%%%%%%%%%%%%%%%%%%%%%%%%%%%%%%%%
%%%%%%%%%%%%%%%%%%%%%%%%%%%%%%%%%% Structure  %%%%%%%%%%%%%%%%%%%%%%%%%%%%%%%%%%
\begin{document}
	\setlength{\headheight}{24.7pt}						% minimum height required by Latexmk
	\pagenumbering{Roman}

%%%%% Title page, quote, declaration, abstract 
	\fancypagestyle{plain}{ 						% no header, no footer
		\fancyhf{} 
		\fancyfoot{}
		\renewcommand{\headrulewidth}{0pt}
		\renewcommand{\footrulewidth}{0pt}
	}	
	\begin{titlepage}
	\includegraphics[width=4cm]{\companyLogo}
	\hfill
	\includegraphics[width=4cm]{\universityLogo}
	\centering
	
	{\large \ \par}
	{\scshape\LARGE \reportType \par}	
	\vspace{1.5cm}
	{\huge\bfseries \thesisTitle \par}
	\vspace{1.5cm}
	{\large \courseOfStudies \\ at \university \par}
	\vspace{1cm}
	{\large \byTitle \par}
	{\Large\itshape \authorName \par}
	\vfill
	{\large \thesisDate \par}
	\vfill
	\begin{tabbing}
		mmmmmmmmmmmmmmmmmmmmmmmmmm			\= \kill 			% define width
		\textbf{\timeOfProjectTitle}			\> \timeOfProject \\
		\textbf{\studentIdTitle, \ \courseTitle}	\> \studentId, \ \course\\
		\textbf{\companyTitle}				\> \company \\
		\textbf{\supervisorTitle}			\> \supervisorName
	\end{tabbing}
\end{titlepage}

% ############################################
\begin{titlepage}
	\noindent
	\begin{minipage}[c][\textheight][c]{\textwidth}
		\centering	
		\textbf{\textit{\quoteText}} \linebreak
		--- \textsc{\quoteAuthor}, \ \textit{\quoteSource}
	\end{minipage}
\end{titlepage}

% ############################################
\chapter*{\declarationTitle}
\declaration
\vspace{4em}

\begin{tabbing}
	mmmmmmmmmmmmmmmmmm	\= \kill 	% define width
	\city, \ \signatureDate 	\> \rule{6cm}{0.4pt} \\
				\> \authorName
\end{tabbing}

% ############################################
\begin{otherlanguage}{ngerman} 
\begin{abstract}
	\abstractGermanText
\end{abstract}
\end{otherlanguage}

% ############################################
\begin{abstract}
	\abstractText
\end{abstract}


%%%%% Table of Contents
	\setlength{\parskip}{0pt}						% after paragraph
	\setcounter{tocdepth}{2}						% depth of table of contents
	\tableofcontents
	\thispagestyle{empty}
		
%%%%% Figures, Tables, Listings	
	\fancypagestyle{plain}{ 
		\fancyhf{} 
		\fancyfoot[R]{\thepage}
		\renewcommand{\headrulewidth}{0pt}
		\renewcommand{\footrulewidth}{0pt}
	}	
	\addtocounter{page}{4}							% increase page number: +4
	% only used acronyms are displayed
% \ac{LOL}  -> First time: Laughing Out Loud (LOL), Else: LOL
% \acs{LOL} -> LOL
% \acf{LOL} -> Laughing Out Loud (LOL)
% \acs{LOL} -> Laughing Out Loud
% \acp{LOL} -> LOLs
% documentation: https://www.ctan.org/pkg/acronym

\chapter*{List of Abbreviations}

\begin{acronym}[AABBCCD]  			% define width of left column 
\setlength{\itemsep}{-\parsep}			% define spacing between acronyms


\acro{LOL}{Laughing Out Loud}
\acro{IMHO}{In My Humble Opinion}
\end{acronym}

	\addcontentsline{toc}{chapter}{\ListOfAbbreviationsTitle}		

	\listoffigures
	\addcontentsline{toc}{chapter}{\ListOfFiguresTitle}	

	%\cleardoublepage
	%\listoftables
	
	\lstlistoflistings
	\addcontentsline{toc}{chapter}{\ListOfListingsTitle}

	\cleardoublepage

%%%%% Content
	\pagenumbering{arabic}
	\setlength{\parskip}{8pt}						% after paragraph
	\renewcommand{\baselinestretch}{1.4}\normalsize				% line height 
	
	\pagestyle{fancy}
	\fancyhf{}
	\renewcommand{\chaptermark}[1]{\markboth{#1}{}}
	\lhead{}
	\rhead{\thechapter  \ -- \ \nouppercase\leftmark}
	\rfoot{\thepage}
	% ----------------------------------------------
\chapter{Structure}\label{ch:figures}
Lorem ipsum dolor sit amet, consectetuer adipiscing elit. Aenean commodo ligula eget dolor. Aenean massa. Cum sociis natoque penatibus et magnis dis parturient montes, nascetur ridiculus mus. Donec quam felis.

% ############################################
\section{Section}\label{sec:images}

In enim justo, rhoncus ut, imperdiet a, venenatis vitae, justo. Nullam dictum felis eu pede mollis pretium. Integer tincidunt. Cras dapibus. Vivamus elementum semper nisi. Aenean vulputate eleifend tellus. Nulla consequat massa quis enim. Donec pede justo, fringilla vel, aliquet nec, vulputate eget, arcu.

\subsection{Subsection}

Phasellus viverra nulla ut metus varius laoreet. Quisque rutrum. Aenean imperdiet. Etiam ultricies nisi vel augue. Curabitur ullamcorper ultricies nisi. Nam eget dui. Etiam rhoncus. Maecenas tempus, tellus eget condimentum rhoncus, sem quam semper libero.

\subsubsection{Subsubsection}
Lorem ipsum dolor sit amet, consectetuer adipiscing elit. Aenean commodo ligula eget dolor. Aenean massa. Cum sociis natoque penatibus et magnis dis parturient montes, nascetur ridiculus mus. Donec quam felis, ultricies nec, pellentesque eu, pretium quis, sem. 

% ----------------------------------------------
\chapter{Text Elements}

% ############################################
\section{New Commands}

This template provides some new commands:

You can add sources in image / listing captions (see \autoref{sec:images}), 

You can create definitions or theorems (see \autoref{sec:maths}).

You can format inline code in a paragraph (see \autoref{sec:code}).

% ############################################
\section{Acronyms and References}

A popular acronym is \acs{LOL}. \acf{IMHO} is another one. 
	
Citations can have page numbers \cite[p. 473]{smi:60}, \cite[pp. 359 - 360]{doe:16}, but they can be left out as well \cite{wai:99}, \cite{mus:16}, \cite{wri:81}, \cite{stu:89}.

You can also add footnotes.\footnote{It is displayed at the bottom of the page.}

\pagebreak

% ############################################
\section{Lists}

Unordered Lists:
\begin{itemize}
\item This is an unordered list. 
\item Item 2.
\item It has three items.
\end{itemize}

Ordered List:
\begin{enumerate}
\item This is an ordered list.
\item Item 2.
\item It has three items.
\end{enumerate}

Ordered List (alphabetical):
\begin{enumerate}[label=\Alph*.]
\item This is an ordered list.
\item Item 2.
\item It has three items.
\end{enumerate}

% ----------------------------------------------
\chapter{Figures}\label{ch:figures}

% ############################################
\section{Images}\label{sec:images}

\autoref{fig:image1} shows how to display images.

\begin{figure}[H]
	\centering
	\includegraphics[width=0.5\textwidth]{example-image-a.pdf}
    \caption{Image}

	\label{fig:image1}
\end{figure}

\begin{figure}[H]
	\centering
	\includegraphics[width=0.5\textwidth]{example-image-b.pdf}
    \caption{Image with Source}
    \captionsource{\cite{mus:16}}	
	\label{fig:image2}
\end{figure}

\begin{figure}[H]
	\centering
	\includegraphics[width=0.5\textwidth]{example-image-c.pdf}
    \caption{Image with Source and Link}
    \captionsource[https://example.org]{J. Doe}	
	\label{fig:image3}
\end{figure}

% ############################################
\section{Diagrams}
\subsection{Sequence Diagram}
\begin{figure}[H]
	\centering
	\begin{sequencediagram}
		\newthread{a}{a: Client}
		\newinst[2]{b}{b: Server 1}
		\newinst[2]{c}{c: Server 2}
		\newinst[2]{d}{d: Database}

		\begin{call}{a}{request}{b}{response}
		\begin{call}{b}{forward}{c}{post processing}
		\begin{call}{c}{request data}{d}{data conversion}
		\end{call}
		\end{call}
		\end{call}
	\end{sequencediagram}
	\caption{Sequence Diagram}
	\label{fig:sequence}
\end{figure}

\subsection{Flow Diagram}

\begin{figure}[H]
	\centering
	\begin{tikzpicture}[node distance=1.6cm]
	\node (start) [startstop] {Start};
	\node (in1) [io, below of=start] {Input 1};
	\node (pro1) [process, below of=in1] {Do something};
	\node (com1) [comment, below of=in1, xshift=-4cm] {STEP 1};
	\node (in2) [io, below of=in1, xshift=5cm] {Input 2};
	\node (sub1) [subprocess, below of=pro1] {\nodepart{two} Subprogram};
	\node (dec1) [decision, below of=sub1, yshift=-1cm] {Decision};
	\node (com1) [comment, below of=dec1, xshift=-4cm, yshift=-1cm] {STEP 2};
	\node (stop) [startstop, below of=dec1, yshift=-1cm] {Stop};

	\draw [arrow] (dec1.west) -- ++(-1,0) node[anchor=south,pos=0.5] {No} |- (sub1.west);
	\draw [arrow] (dec1) -- node[anchor=west] {Yes} (stop);

	\draw [arrow] (start) -- (in1);
	\draw [arrow] (in1) -- (pro1);
	\draw [arrow] (pro1) -- (sub1);
	\draw [arrow] (sub1) -- (dec1);
	\draw [arrow] (in2) -- (pro1);


	\end{tikzpicture}
	\caption{Flow Diagram}
	\captionsource{\cite{mus:16}}
	\label{fig:flow}
\end{figure}

% ----------------------------------------------
\chapter{Conclusion}\label{ch:conclusion}

% ############################################
\section{Mathematics}\label{sec:maths}

\subsection{Definitions}

\begin{define}	
A function $f: X \rightarrow Y$ is injective if and only if for all $x_1, x_2 \in X$, $x_1 \neq x_2$ implies $f(x_1) \neq f(x_2)$.
\end{define}

\subsection{Formulas}

Formulas can be included inline, e.g. $K_E^{CID}$ to $S$ or as an own block.
$$ otp = f_k(i, k), \qquad i = 1, ..., n $$

\pagebreak
% ############################################
\section{Code Listings}\label{sec:code}

Code can be displayed inline, e.g. \code{UPDATE} or \code{factorial(n)}.

Code listings can have numbered lines, captions, syntax highlighting, ... .

\begin{lstlisting}[caption={[Code Without Line Numbers] Code Without Line Numbers \\ \captionsource{\cite{mus:16}}}, label={lst:codeExample}, language=python, numbers=none]
 0  1  2  3  4  5  6  7  8  9 10 11 12 13 14 15 16 17 
 A  B  C  D  E  F  G  H  I  J  K  L  M  N  O  P  Q  R  
\end{lstlisting}

\begin{lstlisting}[caption={Code With Line Numbers}, label={lst:factorial}, language=Java]
public int factorial(int n) {
	if (n == 1) {
		return 1;
	} else {
		return n * factorial(n - 1) ;
	}
}
\end{lstlisting}

	\clearpage

%%%%% Bibliography
	\rhead{\nouppercase\leftmark}
	\printbibliography[heading=bibintoc]
	\clearpage
%%%%% Appendix
	\rhead{A  \ -- \ \nouppercase\leftmark}
	\begin{appendices}
\chapter{Appendix}
\pagebreak

\begin{figure}[H]
	\centering
	 \caption{A Big Image}
	\includegraphics[width=0.95\textwidth]{example-image.pdf}
	\label{fig:appendixImage}
\end{figure}

\begin{lstlisting}[caption={Fibonacci}, label={lst:fibAppendix}, captionpos=t]
public class Fibonacci {
	public static long fib(int n) {
		if (n <= 2) {
			return 1;
		} else {
			return fib(n - 1) + fib(n - 2);
		}
	}

	public static void main(String[] args) {
		for(int i = 1; i < 10; i++) {
			System.out.println(i + " - " + fib(i));
		}
	}
}
\end{lstlisting}
\end{appendices}

\end{document}
